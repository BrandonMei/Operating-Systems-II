\documentclass[onecolumn, draftclsnofoot, 10pt, titlepage, compsoc]{IEEEtran}
\usepackage[utf8]{inputenc}
\usepackage{url}
\usepackage{hyperref}

\title
{%
	Homework 4 \\
	\vspace{0.4cm}
	\large Homework 4 Writeup
	\vspace{0.4cm}
	\large CS444 Spring2018 Group 37
}
\author{Brandon Mei, Brian Huang}
\date{\today}

\begin{document}
\maketitle

\newpage

\section{Implement Design}

At the start, we attempted to approach the problem by looking how the slob first-fit algorithm works or how slob allocator function. When we starting to understand we continue to try to implement the new allocator and observing the default allocator that uses the first-fit algorithm. Which will uses the first page with enough memory. We also research on the best fit meaning and learning that the best-fit algorithm looks through all the possible pages with enough memory space to decide the best option. We tested the between the two algorithms and allocators to see the efficiency. This will need additional of a system call to return the memory usage part. 

\section{Control Log}

\begin{tabular}{l p{5cm} l p{5cm} l}
\textbf{Detail} & \textbf{Author} & \textbf{Description}\\
\href{https://github.com/BrandonMei/Operating-Systems-II/commit/7f8417be8bcac9971c24c53b8cf7e3b8a21fb3bd}{7f8417b} & Brandon Mei & Add testing program for slob file\\\hline
\href{https://github.com/BrandonMei/Operating-Systems-II/commit/9228ec87496add47f3988721fdb898f0cf0fb4b0}{9228ec8} & Brian Huang & Work on the slob file and add original slob with syscall mod\\\hline
\href{https://github.com/BrandonMei/Operating-Systems-II/commit/455b948f6fa4bd664c303a0e7ca383fb4e0bc508}{455b948} & Brandon Mei & update the rest along with write up\\\hline

\end{tabular}

\section{Work Log}

When working on the assignment we first research the task of the assignment is asking us to do and understanding between best-fit and first-fit algorithms for the slob allocator. It took a while to understand the assignment was tasking us to do. This made us to implement the best-fit algorithm with some system calls in the slob.c then change some of the syscall files. It have system calls working with the kernel, We added program to call the system calls and test best-fit algorithm in the kernel. 
\begin{itemize}

\item First figuring out the kernel module and analyze original slob first-fit algorithm
\item analyze original slob first fit algorithm
\item testing the original slob.c
\item add the best-fit algorithm
\item test the new slob.c
\item debug and finishing it up

\end{itemize}

\section{Questions}

\subsection{What do you think the main point of this assignment is?}

We though that the main point of this assignment was to develop a better understanding of memory management and also finding out the fundamentals of the Linux system calls. Learning to utilize them to do request service in the kernel. Another thing we though was the understanding the best fit algorithm was important part of this assignment. By writing a program that computes the first-fit algorithm and best fit algorithm and then compares the fragmentation by each algorithm and additional system calls that will return memory usage.\\

\subsection{How did you personally approach the problem Design decisions, algorithm, etc.}

The way we attempted first was to approach the problem through the trial and error by beginning with a basic slob allocator with only first fit algorithm. Then we will implemented the best fit algorithm system through various online resources, then put the together piece by piece and back tracing when an error occurs. I think the most difficult thing of this assignment was by setting up the linux system calls to reference the memory resources throughout the SLOB allocating. So we began to test with the simple c program that will call a system calls and output the result.\\

\subsection{How did you ensure your solution was correct Testing details, for instance?}

The way we ensure our solution was correct was writing a script that does all the require steps above and set everything up. We tested the Slob allocator with the test program that displayed the best-fit algorithm to show an efficiency of 96 percent and 9 percent of the original first-fit algorithm.\\

\subsection{What did you learn?} 

This assignment made us learned a lot of the Linux slob allocator and the best-fit algorithm. We also learned the linux kernel system calls and modify the memory management allocator. This make us understand the components of the memory management part and the difference between the best-fit and first-fit algorithms.\\

\nocite{*}
\citation{mybib}
\bibliographystyle{IEEEtran}
\bibliography{mybib}

\end{document}