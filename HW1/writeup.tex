\documentclass[onecolumn, draftclsnofoot, 10pt, titlepage, compsoc]{IEEEtran}
\usepackage[utf8]{inputenc}
\usepackage{url}

\title
{%
	Homework 1 \\
	\vspace{0.4cm}
	\large Homework 1 Writeup
	\vspace{0.4cm}
	\large CS444 Spring2018 Group 37
}
\author{Brandon Mei, Brian Huang}
\date{\today}

\begin{document}
\maketitle

\begin{abstract}
The report details the assigned group 37 (Brandon Mei, Brian Huang) to ensure that it properly build the kernel and run it in qemu. The goal is to understand of how the tools and operating within a qemu based Yocto environment and make use of git to all subsequent projects.  
\end{abstract}

\section*{Command Logs}
\begin{tabular}{ | p{5cm} | p{5cm} | }
		\hline
		\textbf{Commands} & \textbf{Explination} \\
		\hline
		\hline
		cd/scratch/spring2018 & Navigates us into the scrach folder. \\
		\hline
		mkdir group37 & Create a folder for our group. \\
		\hline
		cp ../bin/acl\_open./..bin/acl\_open group37 onid(s) & Gives access to other people in thr group \\
		\hline
		cp files to our folder & copying files to our group folder.\\
		\hline
		qemu-system-i386 -gdb tcp::5537 -S -nographic -kernel bzImage-qemux86.bin -drive file=core-image-lsb-sdk-qemux86.ext4,if=virtio -enable-kvm -net none -usb -localtime --no-reboot --append "root=/dev/vda rw console=ttyS0 debug" & Launches qemu.\\
		\hline
		Open a new terminal to run debug commands. & \\
		\hline
		copy kernel files. & \\
		\hline
		gdb -tui & Launch the debug mode to a target remote. \\
		\hline
		target remote: 5537 & Sets the target our port. \\
		\hline
		continue & continues gdb. \\
		\hline
		root & gains access to root. \\
		\hline
		Open another terminal and navigate to the group folder. & \\
		\hline
		Copy conifg files as /.config into our group37 folder. & \\
		\hline
		make -j4 all & Builds the kernel inside the vm. \\
		\hline
		
	\end{tabular} 

\section*{Questions}
\subsection*{What do you think the main point of this assignment is?}
The main point of this assignment was to understand and familiarize with operating within qemu based Yocto environment and know how to build the kernel and run it in qemu properly of the VM. Achieving this requires to have the right command and source the appropriate file prior to building the kernel or running qemu. 
\subsection*{How did you personally approach the problem? Design decisions, algorithms?}
We make sure we have everyone on our group to access our source control repo with assigned group number. Brandon make sure to get the acl open file to work and preceded to source the appropriate file on the environment configuration script in the opt folder. Since our group of two, we split the work evenly among Brandon Mei and Brian Huang. 
\subsection*{How did you ensure your solution was correct? Testing details?}
To ensure our solution was correct by running each of the step one at a time and make sure the result expected to run the source file prior to building the kernel or running qemu by making use of the command. 
\subsection*{What did you learn}
A lesson we learned that is our group had to experience a part of the process to build and see how it runs to the target remote in debug mode. The group learned that kernel boots on the VM and prepare with the next kernel portion of assignments. 

\section*{Control Log}
\begin{tabular}{ | p{5cm} | p{5cm} | p{5cm} | }
	\hline
	commit & Author & commit message \\
	\hline
	\hline
	b5db43c & Brandon Mei & Create README.md \\
	\hline
	d467979 & Brian Huang & added a writeup template \\
	\hline
	1588bdb & Brian Huang & Added some utilitiy files \\
	\hline
	5db36bb & Brian Huang & second test \\
	\hline
	b8a91bd & Brian Huang & Rough draft of the command log \\
	\hline
	7e5da85 & Brian Huang & Roughdraft 2\\
	\hline
\end{tabular}


\section *{Work Log}
\begin{tabular}{ | p{5cm} | p{5cm} | p{5cm} | }
	\hline
	Date & Work Done & By Whom \\
	\hline
	\hline
	April 11, 2018 & Brandon Mei setup the source control repo under our assigned group name. Also setup github for commiting work to thise repo. & Brandon Mei \\
	\hline
	April 12, 2018 & Brandon Mei sets up and builds the kernel. & Brandon Mei \\
	\hline
	April 13, 2018 & Brian Huang set up the documentation for homework 1 and completed the command log. & Brian Huang \\
	\hline
	April 15, 2018 & Brandon Mei finished the answers for the questions. & Brandon Mei \\
	\hline
	April 15, 2018 & Brian Huang Completed the control log and the work log. Working on the final draft. & Brian Huang \\
	\hline
\end{tabular}


\nocite{*}
\citation{mybib}
\bibliographystyle{IEEEtran}
\bibliography{mybib}

\end{document}
